\documentclass{article}
\usepackage{a4wide}
\usepackage{alltt}
\usepackage{amssymb}
\usepackage{amsmath}
\usepackage{latexsym}
\usepackage{stmaryrd}
\usepackage{pig}

\begin{document}
\title{C311 October 2018: Exercise 2}
\author{Neil Ghani and Conor McBride}
\maketitle

\newcommand{\newe}[3]{\mathtt{new}\;#1\;\mathtt{:=}\;#2\;\mathtt{in}\;#3}
\newcommand{\newc}[3]{\mathtt{new}\;#1\;\mathtt{:=}\;#2\;\mathtt{in}\;#3}
\newcommand{\doe}[2]{\mathtt{do}\;#1\;\mathtt{return}\;#2}
\newcommand{\doret}[2]{\mathtt{do}\;#1\;\mathtt{return}\;#2}
\newcommand{\ifel}[3]{\mathtt{if}\;\mathtt{(}#1\mathtt{)}\;#2\;\mathtt{else}\;#3}
\newcommand{\whi}[2]{\mathtt{while}\; \mathtt{(}#1\mathtt{)} \;#2}
\newcommand{\ang}[1]{\langle #1 \rangle}
\newcommand{\NT}[1]{\ang{\mathit{#1}}}
\newcommand{\wk}[2]{#1 \backslash #2}
\newcommand{\evalJ}[4]{#1 \,|\, #2 \Downarrow #3 \,|\, #4}
\newcommand{\evalLam}[2]{#1 \Downarrow #2}
\newcommand{\evalE}[3]{#1 \,|\, #2 \Downarrow #3}

For each of the following, mark each of the claims 1 for True
and 0 for False in the associated Marx-file. While you may collaborate
within your designated groups, you should submit your own answer. Please submit your answer
by Monday 29 October 12pm. Refer to the lecture notes for the various
semantics for IMP and LAM that we have covered in the lecures and
which will be used here.\\[1ex]


\noindent {\bf Part A} Consider IMP-expression $e$ given by $\newe{x}{y + 2}{\doe{x : = x-1}{x + y}}$.
Which of the following claims are true?
\begin{enumerate}
\item If $\sigma$ is $,y := 4$ then $\evalJ{\sigma}{e}{ ,y:= 4, x:=5}{9}$
\item If $\sigma$ is $,y:= 4, z:=5$ then $\evalJ{\sigma}{e}{ ,y:= 4, z:=5}{9}$
\item If $\sigma$ is $,y:= 4, x:=5$ then $\evalJ{\sigma}{e}{ ,y:= 4}{9}$
\item If $\sigma$ is $,z:=5$ then $\evalJ{\sigma}{e}{ ,z:=5}{}$
\end{enumerate}


\noindent {\bf Part B:} Consider the IMP-expression $e$ given by
$$\newe{x}{8}{(\newe{x}{6}{\doe{x := x + y}{x}}) + x}$$. Which of the 
following claims are true?
\begin{enumerate}
\item If $\sigma$ is $,y := 4$ then $\evalJ{\sigma}{e}{,y:=5, x:=8}{18}$
\item If $\sigma$  is $,y := 5, x:=3$ then $\evalJ{\sigma}{e}{,y:=5}{15}$
\item If $\sigma$ is $,y := 15, x:=3$ then $\evalJ{\sigma}{e}{,y:=5,x:=3}{29}$
\item If $\sigma$ is $,y := 15$ then $\evalJ{\sigma}{e}{,y:=5}{29}$
\end{enumerate}


\noindent {\bf Part C:} Consider the LAM-term $two$ given by $\backslash f
\rightarrow \backslash x \rightarrow  f (f x)$
and the lambda term $id$ given by $\backslash x \rightarrow x$. Using
the small step reduction semantics for LAM, which of the following claims are true?

\begin{enumerate}
\item There are no infinite reduction sequences starting from $two
  \;\; id$
\item All reduction sequences from $two \;\; two \;\; id$ eventually end up at $id$
\item The term $two \; (x \; y)$ reduces to $\backslash x \rightarrow
  (x \; y) ((x \; y) \; x)$  
\item The term $two \; two$ reduces to $\backslash f \rightarrow  \backslash x \rightarrow f ( f ( f ( f x)))$
\end{enumerate}



\noindent {\bf Part D:} Using the big-step, call-by-name, reduction
semantics for LAM, and the terms $two$ and $id$ as defned above, which
of the following claims are true?

\begin{enumerate}
\item $\evalLam{two \; id}{\backslash x \rightarrow id \; (id \; x)}$
\item $\evalLam{two \; two \; id}{two \; id}$
\item $\evalLam{3 + two}{5}$
\item $\evalLam{two \; (\backslash x \rightarrow x + x) \; 7}{28}$
\end{enumerate}


\noindent {\bf Part E:} Using the big-step environment semantics for
LAM, which of the following claims are true


%1. ,x:=3 | (\x -> x + x) x \Downarrow ,x:=3 | 12
%2. ,x:=3 | (\x -> x + x) id \Downarrow [,x:=3] id + id
%3. ,x:=3, y:=6 | (\x -> x + y) 5 \Downarrow [,x=3, y:=6] 11
%4. ,x:=3, y:= [] z -> z | (\x -> y x) 5 \Downarrow 5



\begin{enumerate}
\item $\evalJ{,x:=3}{(\backslash x \rightarrow x + x) \; x}{,x:=3}{12}$
\item $\evalE{,x:=3}{(\backslash x \rightarrow x + x) \; id}{[,x:=3]id + id}$
\item $\evalE{,x:=3, y:=6}{(\backslash x \rightarrow x + y) \; 5} {[,x=3, y:=6] 11}$
\item $\evalE{,x:=3, y:= [] z \rightarrow z}{(\backslash x \rightarrow
    y \; x) \;5}{5}$
\end{enumerate}


\newcommand{\execJ}[3]{#1 \,|\, #2 \Downarrow #3}
\newcommand{\cs}{\mathit{cs}}
\newcommand{\cmp}{\mathit{cmp}}
\newcommand{\fbx}[1]{\framebox{\ensuremath{#1}}}



%%%%%%%%%% SMALL STEP RULES
\newcommand{\ssj}[4]{#1|#2 \;\leadsto\; #3|#4}

\noindent {\bf Part F:} We plan to devise a \emph{small step} semantics for IMP, and we need
your help. We have the following judgement forms, following the same metavariable naming
conventions as in the lecture notes.

\begin{itemize}
\item $\ssj{\sigma_0}{e_0}{\sigma_1}{e_1}$\\
  starting with store $\sigma_0$, integer expression $e_0$ steps to store $\sigma_1$ and integer expression $e_1$
\item $\ssj{\sigma_0}{p_0}{\sigma_1}{p_1}$\\
  starting with store $\sigma_0$, boolean expression $p_0$ steps to
  store $\sigma_1$ and boolean expression $p_1$
\item $\ssj{\sigma_0}{c_0}{\sigma_1}{c_1}$\\
  starting with store $\sigma_0$, executing command $c$ steps to store
  $\sigma_1$ and command $c_1$
\item $\ssj{\sigma_0}{\cs_0}{\sigma_1}{\cs_1}$\\
  starting with store $\sigma_0$, executing block $\cs$ steps to store
  $\sigma_1$ and block $\cs_1$
\end{itemize}

The small step semantics should agree with our big step semantics in
the following ways:

\begin{itemize}
  \item For integer expressions, exactly when $\evalJ{\sigma}{e}{\sigma'}{v}$, there should
    be a (possibly empty) sequence of small steps
    $\ssj{\sigma}{e}{\sigma_1}{e_1} \;\;
    \ssj{\sigma_1}{e_1}{\sigma_2}{e_2}\;\;\ldots\;\;
    \ssj{\sigma_n}{e_n}{\sigma'}{v}$
  \item For boolean expressions, exactly when $\evalJ{\sigma}{p}{\sigma'}{b}$, there should
    be a (possibly empty) sequence of small steps
    $\ssj{\sigma}{p}{\sigma_1}{p_1} \;\;
    \ssj{\sigma_1}{p_1}{\sigma_2}{p_2}\;\;\ldots\;\;
    \ssj{\sigma_n}{p_n}{\sigma'}{b}$
  \item For commands, exactly when $\execJ{\sigma}{c}{\sigma'}$, there should
    be a (possibly empty) sequence of small steps
    $\ssj{\sigma}{c}{\sigma_1}{c_1} \;\;
    \ssj{\sigma_1}{c_1}{\sigma_2}{c_2}\;\;\ldots\;\;
    \ssj{\sigma_n}{c_n}{\sigma'}{\{\}}$
  \item For blocks, exactly when $\execJ{\sigma}{\cs}{\sigma'}$, there should
    be a (possibly empty) sequence of small steps
    $\ssj{\sigma}{\cs}{\sigma_1}{\cs_1} \;\;
    \ssj{\sigma_1}{\cs_1}{\sigma_2}{\cs_2}\;\;\ldots\;\;
    \ssj{\sigma_n}{\cs_n}{\sigma'}{}$
\end{itemize}

We are confident in the following rules:

\[\begin{array}{c}
  \Rule{\ssj\sigma e{\sigma'}{e'}}
  {\ssj\sigma{x:=e}{\sigma'}{x:=e'}}
    \qquad
  \Axiom{\ssj{\sigma,x:=v,\wk x{\sigma'}}{x:=v'}{\sigma,x:=v',\sigma'}{\{\}}}    
    \\
  \Rule{\ssj\sigma\cs{\sigma'}{\cs'}}
    {\ssj\sigma{\{\cs\}}{\sigma'}{\{\cs'\}}}
    \\
    \Axiom{\ssj\sigma{\ifel{0}{c_1}{c_0}}\sigma{c_0}}
    \\
    \Rule{\ssj\sigma{e_0}{\sigma\}{e'_0}}}
    {\ssj\sigma{\newe x{e_0}{e_1}}{\sigma'}{\newe x{e'_0}{e_1}}}
    \qquad
    \Axiom{\ssj\sigma{v_0\texttt{+}v_1}\sigma{v_0+v_1}}
\end{array}\]

Note that, in the rule for addition, the step is from the addition of
two values to the value which is their sum: it is not doing nothing!

For each of the following rules, we claim that they are good rules to
add if we want to achieve the correspondence we seek. We want you to
tell us if, in each case, the claim is true.

\[\begin{array}{rc}
    1. & \Axiom{\ssj\sigma{\{\};\cs}\sigma\cs} \\
    2. & \Rule{\ssj\sigma c{\sigma'}{c'}}
         {\ssj\sigma {c;\cs}{\sigma'}{c';\cs}} \\
    3. & \Rule{\ssj\sigma \cs{\sigma'}{\cs'}}
         {\ssj\sigma {c;\cs}{\sigma'}{c;\cs'}} \\
    4. & \Rule{\ssj\sigma{c_1}{\sigma'}{c'_1}}
         {\ssj\sigma{\ifel{p}{c_1}{c_0}}{\sigma'}{\ifel{p}{c'_1}{c_0}}} \\
    5. & \Axiom{\ssj\sigma{\whi pc}\sigma{\{c;\ifel p{\{\whi pc;\}}{\{\}}\}}} \\
    6. & \Axiom{\ssj\sigma{\whi pc}\sigma{\ifel p{\{c;\whi pc;\}}{\{\}}}} \\
    7. & \Axiom{\ssj{\sigma,x:=v,\wk x{\sigma'}}x\sigma v} \\
    8. & \Rule{\ssj\sigma{e_0}{\sigma'}{e'_0}}
            {\ssj\sigma{e_0\texttt{+}e_1}{\sigma'}{e'_0\texttt{+}e_1}} \\
    9. & \Rule{\ssj\sigma{e_1}{\sigma'}{e'_1}}
            {\ssj\sigma{e_0\texttt{+}e_1}{\sigma'}{e_0\texttt{+}e'_1}} \\
    10. & \Rule{\ssj\sigma{e_1}{\sigma'}{e'_1}}
          {\ssj\sigma{v_0\texttt{+}e_1}{\sigma'}{v_0\texttt{+}e'_1}} \\
    11. & \Rule{\ssj{\sigma,x:=e_0}{e_1}{\sigma',x:=e'_0}{e'_1}}
          {\ssj\sigma{\newe x{e_0}{e_1}}{\sigma'}{\newe x{e'_0}{e'_1}}} \\
    12. & \Axiom{\ssj\sigma{\newe xn{n'}}\sigma{n'}}
\end{array}  \]

Of course, there are still plenty of rules missing. We do not claim we
have the full set of small step rules: e.g., we have given no rules
for boolean expressions, so far. It may help to think about what they
should be, and try some examples.

\end{document}